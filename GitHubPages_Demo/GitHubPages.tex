\documentclass[a4paper,11pt]{article}
\usepackage{enumerate}
\usepackage[osf]{mathpazo}
\pagenumbering{arabic}

\begin{document}

\begin{flushright}
Version dated: \today
\end{flushright}
\bigskip
\begin{center}

%Title
\noindent{\Large \bf A quick GitHub pages tutorial}\\
\bigskip
%Author
\noindent { Thomas Guillerme - guillert@tcd.ie - http://tguillerme.github.io/}\\

\end{center}

\raggedright
 \pagestyle{empty}
\pagenumbering{gobble}

This is a really quick tutorial on how to create GitHub pages, whether you want them for a \textbf{project page} (i.e. specific to one of your projects) or just a \textbf{home page} (i.e. just stuff about you're wonderful life).

\section{Generating the page}
Dead easy, just follow these steps:
\begin{enumerate}
\item{Register on GitHub:}\\
https://github.com/

\item{\textit{Download a GUI client (recommended):}}\\
\begin{itemize}
\item{Windows: https://windows.github.com/}
\item{Mac: https://mac.github.com/}
\item{Linux/Nerd: You're grant, it's really easy on terminal too.}
\end{itemize}

\item{Generate your page:}\\
https://pages.github.com/
\end{enumerate}

\section{Editing the page}
Ones every step is completed, let the magic happen and just modify content depending on your mood on your computer. Then commit/push it on GitHub and it will instantly update your online content.

GitHub pages are written in html language, if you're familiar with it, just have fun, if not here is a non-exhaustive list of resources:
\begin{itemize}
\item{Trial and error, get your hands dirty!}
\item{http://www.w3schools.com/}
\item{http://www.htmldog.com/guides/html/beginner/}
\item{http://html.net/}
\end{itemize}

It's not much more complicated than that!
\end{document}
